\documentclass[a4paper,10pt]{article}
\usepackage[utf8]{inputenc}
\usepackage{geometry}
\geometry{a4paper, margin=1in}
\usepackage{enumitem}
\usepackage{hyperref}
\usepackage{amsmath}

\usepackage{helvet}

\pagestyle{empty}

\begin{document}

%======================
% Name & Contact Information
%======================
\begin{center}
    {\LARGE \textbf{JOÃO GUILHERME ALVES SANTOS}} \\
    \vspace{0.2cm}
    \href{mailto:guilherme-alves16@usp.br}{guilherme-alves16@usp.br}

\end{center}

%======================
% EDUCATION
%======================
\section*{EDUCATION}
\vspace{-1.5em} % Adjust vertical space
\noindent\rule{\textwidth}{0.4pt} % Horizontal line

\noindent\textbf{University of São Paulo}, São Paulo, Brazil \hfill  \textit{Apr 2021 -- Dec 2024} \\
\textit{Bachelor's in Computer Science} \hfill GPA: 9.2/10 \\
The Computer Science program is a four-year course that offers great flexibility in selecting courses. It features a system of specialization tracks, including Theoretical Computer Science, in which I have specialized. \\

%\vspace{0.2cm}

%======================
% PROJECTS AND EXPERIENCE
%======================
\section*{PROJECTS AND EXPERIENCE}
\vspace{-1.5em} % Adjust vertical space
\noindent\rule{\textwidth}{0.4pt} % Horizontal line

\noindent\textbf{Undergrad Research Project}  \hfill  \textit{Jan 2024 -- Dec 2024} \\
A scientific research under the supervision of Profa. Cristina Gomes Fernandes and funded by FAPESP in the 
process number 2023/16197-0. Is this project we work on approximation algorithms and inapproximability results for three NP-hard clustering problems: $k$-center, $k$-median and facility location. The approximation algorithms examined employ a variety of methods, including the greedy approach, local improvement, dual fitting, deterministic rounding, probabilistic rounding, and the primal-dual method.\\

\noindent\textbf{Teaching Assistant}  \hfill  \textit{Aug 2023 - Dec 2023}\\
I was a Teaching Assistant in Automata, Computability, and Complexity. As a TA, I assist students in navigating the complexities of theoretical computer science, ranging from finite automata to computational complexity theory. \\

\noindent\textbf{Teaching Assistant} \hfill  \textit{Aug 2022 - Dec 2022}\\
I was a Teaching Assistant in Principles of Algorithm Design. As a TA, I support students in mastering foundational data structures and algorithms, including stacks, queues, linked lists, recursion, sorting, and search. I provide guidance on algorithm design, correctness proofs, and performance analysis to build problem-solving skills and prepare students for more advanced topics in computer science.\\

\vspace{0.2cm}
%======================
% PARTICIPATION IN EVENTS
%======================
\section*{PARTICIPATION IN EVENTS AND COURSES}
\vspace{-1.5em} % Adjust vertical space
\noindent\rule{\textwidth}{0.4pt} % Horizontal line

\noindent\textbf{International Symposium on Scientific and Technological Initiation of USP} \hfill \textit{22, 23 Oct 2024}\\
The International Symposium on Scientific and Technological Initiation of the University of São Paulo (SIICUSP) is an annual event that highlights research findings from undergraduate students at USP. The project ``Approximation Algorithms for Clustering Problems" was present at the 32º SIICUSP, under the supervision of Profa. Cristina Gomes Fernandes. \\

\noindent\textbf{6th São Paulo Workshop on Optimization, Combinatorics, and Algorithms} \hfill \textit{9 - 13 Oct 2024}\\
The São Paulo Workshop on Optimization, Combinatorics, and Algorithms (WoPOCA) is an event that aims to create a collaborative environment for discussing open problems in optimization, combinatorics, and algorithms, involving professors, graduate, and undergraduate students. Worked mainly in the Balanced Connected \(k\)-Partition Problem for planar graphs. The student also participates in the mini-course 'Matroids and Greedy Algorithms' ministred by Prof. Orlando Lee. \\


\noindent\textbf{ICPC Summer School} \hfill \textit{24 Jan - 5 Feb 2022}\\
The ICPC Summer School is an intensive program focused on competitive programming and algorithmic problem-solving. I gained hands-on experience in advanced algorithm design, data structures, and optimization techniques.
\\
%======================
%HIGH SCHOOL ACADEMIC ACHIEVEMENTS
%======================
% \section*{ACADEMIC ACHIEVEMENTS}
% \vspace{-1.5em} % Adjust vertical space
% \noindent\rule{\textwidth}{0.4pt} % Horizontal line
% \begin{itemize} 
% \end{itemize}


\end{document}
